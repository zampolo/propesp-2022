\documentclass[a4paper,10pt]{article}

\renewcommand{\familydefault}{\sfdefault} %font selection \ttdefault, \sfdefault or \rmdefault
%=== preambulo === 
\usepackage[brazilian]{babel}
\usepackage{color,enumerate}
\usepackage{graphicx}
%\usepackage[toc,page]{appendix}

\usepackage[table]{xcolor}
\definecolor{lightgray}{gray}{0.9}
\definecolor{lightblue}{RGB}{153,204,255}
\definecolor{lightgreen}{RGB}{204,255,229}
\definecolor{lightyellow}{RGB}{255,255,153}
\definecolor{lgray}{gray}{0.7}

\usepackage{booktabs}
\usepackage[T1]{fontenc}

\usepackage[utf8]{inputenc}
\usepackage{subfigure}
\usepackage{fullpage}
\usepackage[affil-it,auth-sc]{authblk}
\usepackage[backend=biber, style=authoryear, doi=false, isbn=false, url=false]{biblatex}
\addbibresource{bib/actrec.bib}

\renewcommand\Authand{ e }
\renewcommand\Authands{, e }

\date{Março de 2022}%Julho 


\title{Sistemas de segurança e proteção em ambientes industriais baseados em técnicas de visão computacional e aprendizagem profunda}

\author[1]{Ronaldo de Freitas Zampolo}
%\author[2]{Adriana Rosa Garcez Castro}
%\author[2]{Agostinho Luiz da Silva Castro}

\affil[1]{Laboratório de Processamento de Sinais\\
Instituto de Tecnologia\\
Universidade Federal do Pará\\
66075-110 Belém, PA}

%\affil[2]{Laboratório de Monitoramento Inteligente e Soluções em Telecomunicações\\
%Instituto de Tecnologia\\
%Universidade Federal do Pará\\
%66075-110 Belém, PA}

\begin{document}
\maketitle
\newpage\tableofcontents\newpage

\section{Identificação do projeto}

\begin{table}[!th]
\begin{tabular}{l p{0.7\textwidth}}
%\hline
\textbf{Título do projeto:}           & Sistemas de visão computacional baseados em aprendizagem profunda para reconhecimento de atividades humanas\\
\textbf{Grande área de conhecimento:} & Engenharias.\\
\textbf{Área de conhecimento:}        & Engenharia Elétrica.\\
\textbf{Sub área:}                    & Telecomunicações.\\
\textbf{Instituição:}                 & Universidade Federal do Pará (UFPA)\\
\textbf{Unidade:}                     & Instituto de Tecnologia (ITEC)\\
\textbf{Sub-unidade:}                 & Faculdade de Engenharia da Computação e Telecomunicações (FCT)\\
\textbf{Endereço profissional:}       & Laboratório de Processamento de Sinais (LaPS)\\
                                     & Anexo do LEEC, altos, sala 32\\
                                     & Av. Augusto Correa, 1\\
                                     & 66070-110 Guamá Belém, PA, Brasil\\
\textbf{Telefones:}                    & +55 91 3201-7674 (LaPS), +55 91 98119-8840\\
\textbf{Emails:}                      & laps@ufpa.br, zampolo@ufpa.br\\
\textbf{Coordenador do projeto:}      & Ronaldo de Freitas Zampolo \\
\textbf{Outras instituições participantes:} & Instituto SENAI de Inovação -- Tecnologias Minerais (ISI-TM) \\
%			      & Adriana Rosa Garcez Castro\\
%			      & Agostinho Luiz da Silva Castro\\
%\textbf{Cargo atual}          & Professor associado\\
%\textbf{Data de nascimento}   & 10 de fevereiro de 1973\\
%\textbf{Local de nascimento}  & São Paulo, SP, Brasil\\
%\textbf{\slang{Endereço residencial}{Home address}}  & Av. Pedro Miranda, 1929\\
%                                		     & Ed. Pollux, ap 602\\
%                                		     & 66085-024 Pedreira Belém, PA, \slang{Brasil}{Brazil}\\
%                              & +55 91 3244-3578 (\slang{res.)}{home)}\\
%                              & +55 91 8119-8840 (celular)\\
%	                      & zampolo@ieee.org\\
\textbf{Período de execução:}  & 24 meses\\
%\textbf{Supervisor}             & Prof.~Patrick Le Callet\\
%                                & École Polytechnique de Nantes\\
%                                & Université de Nantes\\
%                                & IRCCYN/IVC\\
%                                & Rue Christian Pauc La Chantrerie\\
%                                & BP 50609 44306 Nantes Cedex 3 \slang{França}{France}\\
%\hline
\end{tabular}
\end{table}
\newpage
%----------------------------------------------------------------------------



%\begin{abstract}
%\input{resumo}
%\end{abstract}
%===
\section{Equipe do projeto}
\label{sec:equip}
% matrícula, nome completo, tipo, titulação máxima, unidade, função no projeto, carga horária
% tipo: ta, técnico administrativo; pv: professor visitante; pe: professor permanente (unidade do projeto); pp: professor participante (outra unidade); ppe: professor participante externo; te: técnico administrativo externo; pb: professor bolsista de agência de fomento
% função no projeto: cd: coordenador; cl: colaborador; cs: consultor
%Ana (ISI), Carol (LaPS), Bruno (ISI), Agostinho ?, Adriana ?
Alunos de graduação (2, pivic, tcc)

%===

\section{Introdução}
\label{sec:intro}
%\include{introducao}
%===

\section{Justificativa}
\label{sec:just}
%\include{justificativa}
%===

\section{Objetivos}% do projeto de pesquisa}
\label{sec:obj}
%\include{objetivos}
%===

\section{Metodologia}
\label{sec:metod}
%\include{metodo}
%===

\section{Metas}
\label{sec:metas}
%\include{metas}
%===

\section{Cronograma de Atividades}
\label{sec:cronos}
%\include{cronos}
%===

%\section{Orçamento}
%\label{sec:orc}
%\input{orcamento}
%===


%\section{Referências}
%\label{sec:ref}
\printbibliography
%===

%\section{Resumo do projeto}%Breve descrição do problema}
%\label{sec:descr}
%\include{descricao}
%===                                          

%\section{Revisão da literatura científica}
%\label{sec:rev}
%\input{revisao}
%===                                          

%\section{Trabalho preparatório}
%\label{sec:prep}
%\input{preparatorio}
%\newpage
%===                                          

%\section{Elementos do projeto}
%\label{sec:proj}
%\input{proj}

\end{document}
